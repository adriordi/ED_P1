\section{Introduccion}

En esta primera práctica se ha analizado la eficiencia de diversos algoritmos tanto empíricamente como teoricamente, ademas se ha procedido a comprobar como se comportan los algoritmos en diversos casos, tanto extremos (mejor y peor caso) comop evaluando como afecta el entorno, el proceso de comilacion, etc. a las implementaciones y eficiencia de estos en el mundo real. Los algoritmos evaluados han sido los siguientes:

\begin{itemize}
\item Busqueda lineal.
\item Ordenacion lineal o algoritmo burbuja.
\item Busqueda binaria.
\item Multiplicacion de matrices.
\item Ordenacion por mezcla o mergesort.
\end{itemize}

Salvo que en algun apartado de este texto se indique lo contrario todos los algoritmos descritos a continuación se han compilado con el compilador g++ para maquinas de 64 bits, y ejecutado en la maquina descrita a continuación:

Lenovo ThinkPad T440p.

\begin{lstlisting}[language=bash]

CPU:       Dual core Intel Core i5-4210M (-HT-MCP-) cache: 3072 KB 
clock speeds: max: 3200 MHz 1: 2647 MHz 2: 2664
MHz 3: 2589 MHz 4: 2608 MHz

Memoria:

MemTotal:        7864344 kB
MemFree:          438576 kB
MemAvailable:    1471416 kB
Buffers:          242820 kB
Cached:          1421444 kB
SwapCached:            0 kB

\end{lstlisting}

Nota: Los ejercicios descritos en este documento no mantienen el mismo orden que en la presentación de la práctica, ya que cuando se comience a hablar de un algoritmo se detallaran todos los analisis y pruebas realizadas, no obstante antes de cada analisis se detallara el ejercicio en concreto del que se trata.